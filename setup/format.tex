% !Mode:: "TeX:UTF-8" 

% 备注
\long\def\zhao#1{\textcolor{red}{\bf **#1**}}

%========================================= Font ==========================================
% 定义字号
\newcommand{\yihao}{\fontsize{26pt}{26pt}\selectfont}       % 一号, 1.倍行距
\newcommand{\xiaoyi}{\fontsize{24pt}{24pt}\selectfont}      % 小一, 1.倍行距
\newcommand{\erhao}{\fontsize{22pt}{22pt}\selectfont}		% 二号, 1.倍行距
\newcommand{\xiaoer}{\fontsize{18pt}{18pt}\selectfont}      % 小二, 单倍行距
\newcommand{\sanhao}{\fontsize{16pt}{16pt}\selectfont}      % 三号, 1.倍行距
\newcommand{\xiaosan}{\fontsize{15pt}{15pt}\selectfont}     % 小三, 1.倍行距
\newcommand{\sihao}{\fontsize{14pt}{14pt}\selectfont}       % 四号, 1.倍行距
\newcommand{\xiaosi}{\fontsize{12pt}{12pt}\selectfont}      % 小四, 1.倍行距
\newcommand{\wuhao}{\fontsize{10.5pt}{10.5pt}\selectfont}   % 五号, 单倍行距
\newcommand{\xiaowu}{\fontsize{9pt}{9pt}\selectfont}        % 小五, 单倍行距

% 默认字体
\makeatletter
\renewcommand\normalsize{
	\@setfontsize\normalsize{12pt}{12pt}
	\setlength\abovedisplayskip{8pt}
	\setlength\abovedisplayshortskip{8pt}
	\setlength\belowdisplayskip{\abovedisplayskip}
	\setlength\belowdisplayshortskip{\abovedisplayshortskip}
	\let\@listi\@listI
}

% 设置行距和段落间垂直距离
\setlength{\parindent}{2em}
\setlength{\parskip}{3pt plus1pt minus1pt}
\def\defaultfont{\renewcommand{\baselinestretch}{1.5}\normalsize\selectfont}

% 设置字间距,使每行37个字,若要减少每行字数(一般以34个),将0.56pt的值增加
\renewcommand{\CJKglue}{\hskip 0.56pt plus 0.08\baselineskip}

% 公式跨页设置,公式之前可以换页,公式出现在页面顶部
\predisplaypenalty=0
\allowdisplaybreaks[4]

%========================================= TOC ===========================================
% 中文目录格式定义
\renewcommand\contentsname{目\quad 录}
\titlecontents{chapter}[3.8em]{\hspace{-3.8em}}{\thecontentslabel~~}{}{\titlerule*[4pt]{.}\contentspage}
\dottedcontents{section}[38pt]{}{22pt}{0.3pc}
\dottedcontents{subsection}[70pt]{}{32pt}{0.3pc}

% 英文目录格式定义: 细点\@dottedtocline  粗点\@dottedtoclinebold
\renewcommand*\l@chapter{\@dottedtocline{0}{0em}{5em}}
\renewcommand*\l@section{\@dottedtocline{1}{1em}{1.8em}}
\renewcommand*\l@subsection{\@dottedtocline{2}{2.9em}{2.5em}}
\def\@dotsep{0.75}           % 定义英文目录的点间距
\setlength\leftmargini {50pt}
\setlength\leftmarginii {50pt}
\def\engcontentsname{CONTENTS}
\newcommand\tableofengcontents{
	\@restonecolfalse
	\chapter*{\engcontentsname  %chapter*上移一行,避免在toc中出现。
		\@mkboth{%
			\engcontentsname}{\engcontentsname}}
	\@starttoc{toe}%
	\if@restonecol\twocolumn\fi
}

%======================================= Chapter =========================================
% 章节格式定义
\CTEXsetup[number={\arabic{chapter}}]{chapter}
\renewcommand\chaptername{\thechapter}
\CTEXsetup[name={,}]{chapter}
\setcounter{secnumdepth}{4} \setcounter{tocdepth}{2}
\titleformat{\chapter}{\center\sanhao}{\chaptertitlename}{0.5em}{}
\titlespacing{\chapter}{0pt}{-5mm}{5mm}
\titleformat{\section}{\xiaosan}{\thesection}{0.5em}{}
\titlespacing{\section}{0pt}{3mm}{3mm}
\titleformat{\subsection}{\sihao}{\thesubsection}{0.5em}{}
\titlespacing{\subsection}{2.13em}{3mm}{3mm}

% 重新定义BiChapter命令,可实现标题手动换行,但不影响目录
\def\BiChapter{\relax\@ifnextchar [{\@BiChapter}{\@@BiChapter}}
\def\@BiChapter[#1]#2#3{\chapter[#1]{#2}
    \addcontentsline{toe}{chapter}{\xiaosi \thechapter\hspace{0.5em} #3}}
\def\@@BiChapter#1#2{\chapter{#1}
    \addcontentsline{toe}{chapter}{\xiaosi \thechapter\hspace{0.5em}{\boldmath #2}}}
% 定义双标题
\newcommand{\BiSection}[2]{
	\section{#1}
	\addcontentsline{toe}{section}{\protect\numberline{\csname thesection\endcsname}#2}
}
\newcommand{\BiSubsection}[2]{
	\subsection{#1}
	\addcontentsline{toe}{subsection}{\protect\numberline{\csname thesubsection\endcsname}#2}
}
\newcommand{\BiSubsubsection}[2]{
    \subsubsection{#1}
    \addcontentsline{toe}{subsubsection}{\protect\numberline{\csname thesubsubsection\endcsname}#2}
}

% 该附录命令适用于有章节的完整附录和用于发表文章和简历的附录
\newcommand{\BiAppChapter}[2]{
	\phantomsection 
	\chapter{#1}
	\addcontentsline{toe}{chapter}{\xiaosi Appendix \thechapter~~#2}
}
\newcommand{\BiAppendixChapter}[2]{
	\phantomsection
	\markboth{#1}{#1}
	\addcontentsline{toc}{chapter}{\xiaosi #1}
	\addcontentsline{toe}{chapter}{\xiaosi #2}  \chapter*{#1}
}



%======================================== Header =========================================
% 定义页眉和页脚
\newcommand{\makeheadrule}{
\rule[8pt]{\textwidth}{0.4pt} \\[-22pt]
\rule{\textwidth}{0.4pt}}
\renewcommand{\headrule}{
    {\if@fancyplain\let\headrulewidth\plainheadrulewidth\fi
     \makeheadrule}}
\pagestyle{fancyplain}

% 不要注销这一行,否则页眉会变成:“第1章1  绪论”样式
\renewcommand{\chaptermark}[1]{\markboth{\chaptertitlename~\ #1}{}}
\fancyhf{}
\fancyhead[CO]{\wuhao\leftmark}
\fancyhead[CE]{\wuhao 西安交通大学本科毕业设计(论文)}%
\fancyfoot[LE,RO]{\wuhao\thepage}

%======================================= Theorem =========================================
% 定理环境格式定义
\theoremstyle{plain}
\theorembodyfont{\rmfamily}
\theoremheaderfont{\hei\rmfamily}
\setlength{\theorempreskipamount}{8pt}
\setlength{\theorempostskipamount}{8pt}
\theoremseparator{:}
\theoremsymbol{$\Diamond$} \newtheorem{definition}{\hskip 2em \hei 定义}[chapter]
\theoremsymbol{$\blacklozenge$} \newtheorem{example}{\hskip 2em \hei 例}[chapter]
\theoremsymbol{$\square$} \newtheorem{theorem}{\hskip 2em \hei 定理}[chapter]
\theoremsymbol{$\blacksquare$} \newtheorem*{proof}{\hskip 2em \hei 证明}

%========================================= FTE ===========================================
% 图表公式的编号为1-1格式,子图编号为 a) 的格式
\renewcommand{\thefigure}{\arabic{chapter}-\arabic{figure}}
\renewcommand{\thesubfigure}{\alph{subfigure})}
\renewcommand{\p@subfigure}{\thefigure~}
\renewcommand{\thetable}{\arabic{chapter}-\arabic{table}}%使表编号为 7-1 的格式
\renewcommand{\theequation}{\arabic{chapter}-\arabic{equation}}%使公式编号为 7-1 的格式

% 定制浮动图形和表格标题样式
\captionnamefont{\wuhao}
\captiontitlefont{\wuhao}
\captiondelim{~~}
\hangcaption
\renewcommand{\subcapsize}{\wuhao}
\setlength{\abovecaptionskip}{0pt}
\setlength{\belowcaptionskip}{0pt}

% 关于公式的几个定义
\renewcommand{\Re}{\mathrm{Re}}
\renewcommand{\Im}{\mathrm{Im}}
\newcommand{\mbf}[1]{\mathbf{#1}}
\newcommand{\Exp}{\mathrm{E}}
\newcommand{\dif}{\mathrm{d}}
\newcommand{\seq}[2]{#1_1,#1_2,\cdots,#1_#2}
\newcommand{\iprod}[2]{\langle #1,#2 \rangle}

% 调整罗列环境的布局
\setitemize{leftmargin=3.23em,itemsep=0em,partopsep=0em,parsep=0em,topsep=-0em}
\setenumerate{leftmargin=3.45em,itemsep=0em,partopsep=0em,parsep=0em,topsep=0em}

% 自定义项目列表标签及格式 \begin{publist} 列表项 \end{publist}
\newcounter{pubctr} %自定义新计数器
\newenvironment{publist}{
	\begin{list}{[\arabic{pubctr}]}{
     \usecounter{pubctr}
     \setlength{\leftmargin}{2em}     % 左边界
     \setlength{\labelsep}{1em}       % 标号和列表项之间的距离,默认0.5em
     \setlength{\parsep}{0ex}         % 段落间距
     \setlength{\itemsep}{0ex}        % 标签间距
    }}
{\end{list}}%

% 更改算法标题
\renewcommand{\algorithmcfname}{算法}
\renewcommand{\algocf@captiontext}[2]{\wuhao#1\algocf@typo ~ \AlCapFnt{}#2}
\renewcommand\thealgocf{\csname the\algocf@within\endcsname-\@arabic\c@algocf}
\SetAlCapSty{textrm}
\SetAlCapSkip{1ex}

% 更改代码标题
\renewcommand{\lstlistingname}{代码}
\renewcommand{\thelstlisting}{\arabic{chapter}-\arabic{lstlisting}}

%======================================== Other ==========================================
% 避免宏包 hyperref 和 arydshln 不兼容带来的目录链接失效的问题。
\def\temp{\relax}
\let\temp\addcontentsline
\gdef\addcontentsline{\phantomsection\temp}
\gdef\hitempty{}

\renewcommand\frontmatter{\cleardoublepage
	\@mainmatterfalse
	\pagenumbering{Roman}
}

% 引用改为上标
\newcommand{\upcite}[1]{\textsuperscript{\textsuperscript{\cite{#1}}}}

\makeatother
